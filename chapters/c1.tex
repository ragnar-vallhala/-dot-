\chapter{The Entry}
It was the first day after summer vacation, and June still clung to the air like a heavy blanket. The fans on the classroom ceiling dragged themselves in slow circles, pushing warm air from one corner to another. Shirts stuck to backs, ties tightened for morning assembly still refused to loosen their grip on sweaty necks. Seventh graders filled the room with chatter. A month apart felt like a year at their age. Everyone had stories to tell and voices seemed to rise from every bench—except the last one, where Shiva sat with his usual companions, Vijay and Rajveer. The three of them—self-declared Tridev—leaned close, talking as if the rest of the class didn’t exist.

“So, where did you go this time?” Shiva asked, pulling at his tie until the knot finally gave up. Vijay grinned. “To my mama’s place. He has a big fishpond now. I even went with him for a fish catch. They use those huge nets, you know? One pull and you get hundreds. He’s doing well—government job, side business… everything.” Rajveer nodded. “Fisheries is good money. My uncle also has ponds. People earn a lot.” Vijay puffed up a little. “And my elder uncle might become village head this time. He even lets villagers fish for free sometimes. Gains goodwill that way.” Shiva listened, nodding, though something tight curled in his chest. He had nothing grand like this to share—no ponds, no wealthy relatives, no election-bound uncles. That small pinch of insignificance, the kind he never admitted to anyone, settled quietly inside him. So he reached for the one thing he did have—knowledge. “I read that the economy’s improving,” he said. “Lot of privatization now. The government is pushing business growth.” Vijay snapped his fingers. “Oh! I also read about INS Vikramaditya. India got it from Russia. It’s massive! Like a whole city floating.” “What does it do?” Rajveer asked. “It carries fighter jets,” Vijay said dramatically. “Just imagine—planes taking off from the sea!”

Before anyone could reply, the classroom door swung open. Like a wave, every student scrambled back to their seats. Books snapped shut, whispers died mid-sentence. “Good morning, Ma’am!” the class chorused. Ms. Sneha entered with her attendance register tucked under her arm. Her smile was warm—the kind that made even the boring subjects feel interesting. Shiva liked her classes. She explained Social Science like it was some story unfolding around them, not just printed text to memorize. She began calling out names. One by one, students stood and answered. “Present, Ma’am.” “Present, Ma’am.” The rhythm was familiar, almost comforting, even with fifty-plus students in the room. Shiva wasn’t paying much attention—until a soft voice interrupted the routine.

“My name is Gauri Mishra. I studied previously in the city.” The sound penetrated Shiva's ears like arrows piercing through gut. Such soft sooth tone, ``who is she?'' he thought. His head lifted before he even realized it. She stood in the second row, first bench, her chin slightly raised as if she wasn’t entirely sure she wanted to be here. Pink frameless spectacles rested neatly on her nose. Two ponytails fell over her shoulders, tied tight and perfectly even. Something about her speech—clear, calm, practiced—made the whole room seem quieter to Shiva. The heat, the fans, even Vijay’s previous excitement… everything faded a little. He kept watching. Not staring, exactly—just drawn in, as if she were a new word in a book he wanted to understand. The rest of the class buzzed on, but Shiva barely heard any of it. The period ended quicker than he wanted. When the bell rang, he almost jolted upright. “I’ll be back,” he muttered to his friends and immediately walked toward the front benches. He didn’t dare speak to Gauri. Even looking at her too long felt like breaking some rule. Instead, he stopped beside Divya, who sat next to her.

“Hi, Divya. How are you?” Divya raised an eyebrow. “So you remembered I exist?” Shiva laughed awkwardly. “Hey—those two kept talking. I meant to come earlier.” Her teasing softened into a smile. She clearly understood why he had suddenly found the front benches so interesting. On the second bench, Mohit sat hunched over his notebook. Shiva tapped the desk. “Mohit, can you move a bit? I need to sit here.” Mohit brightened instantly. “Sure!” Shiva sat down behind Gauri. Not close enough to draw attention—just close enough to see the tiny strands of hair escaping her ponytail, the slight tilt of her head when she read something, the city-girl confidence in the way she straightened her posture. For the first time that day, the heat didn’t bother him anymore.

The next period was Science. The moment Mr. Arvind Rai stepped into the room, the class rose again in a messy chorus. “Good morning, sir!” Shiva usually brightened at the sight of Mr. Arvind. Science was his subject—his territory. But today his mind was elsewhere, drifting around the first bench where Gauri sat, her ponytails perfectly still even in the warm breeze from the windows. He didn’t even notice the commotion until a sharp voice cracked through his thoughts. “Shiva! You, too, haven’t completed the homework?” The class froze. Mr. Arvind’s eyes were fixed on him, brows pulled tight. Shiva blinked. Around him, several students were standing with notebooks open. His stomach dropped. “Were you playing all summer?” Mr. Arvind demanded. “Why is your homework incomplete?” “No, sir!” Shiva shot up immediately. “I—I did my homework. All of it. Even made the diagram on chart paper. It's... in my bag.” A ripple of murmurs spreads. Someone snickered. And just for a moment, as Shiva glanced ahead, he caught Gauri turning slightly—her lips curled in a quiet, teasing smile. A smile meant to mock him. Yet it somehow made his heart jump. “So you forgot to bring your bag?” Mr. Arvind’s voice softened into a mocking tone. “No, sir! It’s just… I changed my seat.” Shiva hurried across the room to the last bench, fumbling with his bag. He pulled out the neatly finished homework and the chart paper. Mr. Arvind followed, checked the work, then signed it. “Since you’ve changed your seat, sit here now,” he ordered, pointing to the last bench he had his bag on. “Yes, sir,” Shiva replied, settling down with heat rising in his cheeks. He kept his gaze low. Great first impression, he thought bitterly. What must she think now? Some careless boy who forgets his homework? I’m the topper… she won’t know that. He risked a tiny glance at her. But she was already looking ahead, attentive, ready for the lesson. Whatever expression she’d had earlier was gone, replaced by calm focus.

Mr. Arvind began explaining the chapter on the water cycle. Pages flipped. Pens clicked. Slowly, Shiva’s embarrassment loosened its grip as the lesson pulled him in. Water moving through clouds and mountains… traveling for centuries… shaping life, shaping earth. He imagined rivers carving their way through stone, raindrops falling before humans ever existed. The thought soothed him—water had patience, purpose. Maybe he could too.

By the time the bell rang after four periods, the classroom transformed instantly. The “recess bell” didn’t ring—the class exploded. Benches rattled as kids shot up like springs. Some rushed to the washroom, some sprinted to the playground, and some clustered around the lunches. Shiva didn’t bring lunch anymore. Carrying a tiffin felt childish, like something kids in lower classes still did. Instead, he joined the familiar wandering tribe—those who survived on bites stolen from friends’ lunchboxes. He followed Vijay and Rajveer, hopping booth-to-booth like food pirates. Aryan’s lunchbox was the biggest treasure. He always brought the good stuff—maggi, sandwiches, aloo-paratha. Naturally, it disappeared within seconds. Aryan stared at the empty box like a man betrayed. “I swear I’m not bringing lunch from tomorrow,” he muttered. “I barely get two bites of my own food.” From the side, Suraj grinned. “Why? Is suji running out in your shop?” Laughter burst across the group. Even Shiva cracked up as they drifted toward the next target. When the raid was over, Suraj opened his own steel lunchbox with exaggerated hope—only for the group to collectively groan. “Seriously? Karela?” Aryan recoiled. “Who brings bitter gourd on the first day? Want us to celebrate with poison?” Suraj shrugged happily, already chewing. “I don’t have a suji shop like you, Aryan. We have this. Eat if you want—I don’t stop anyone.” Aryan walked away with a smug look on his face. Shiva wiped tears of laughter from his eyes. 

Just then, Vijay slid back into the room, drifting on the dusty floor as if performing some stunt. “Let’s go to the ground!” he declared. “Isn’t it boiling hot outside?” Aryan asked. Rajveer smirked. “Why scared? Sun will burn you? Don’t be like girls hiding from the heat.” With that unbeatable logic, the boys exchanged glances—then gave in. Shiva, Vijay, Aryan, and Rajveer headed out into the sunlit corridor, laughter trailing behind them like shadows stretching in the afternoon light.

The final bell rang, and just like that, the day dissolved into the familiar rush of feet and voices. Shiva walked out with the seventh-grade boys, a strange mix of joy and nervous energy pulsing through him. He had met his friends again, shared laughter, stolen lunch bites—but Gauri’s face kept flashing before his eyes like a sudden spark. Every time the memory surfaced, a smile slipped onto his face without warning. But school was over now. Time to go home. Two long columns—girls on one side, boys on the other—began marching down the school street like Roman legions called to formation. Teachers stood at intervals, trying to hold the fragile order together. Younger classes led the way, followed by the older ones, drifting forward like slow-moving waves. School vans filled one after another, swallowing lines of students. The quiet discipline of the morning was replaced with noise from every direction—shouts, laughter, horns, sandals scraping the dusty ground. Ms. Sneha finally instructed the seventh-grade girls to move, then the boys. Shiva joined the tail end of the line.

He climbed into the school van—“the magic,” as everyone called it—and slipped into the back seat, the one that could fit three. It was his favorite spot. From there he could see the road stretching behind them, and it was where Krishna, a ninth grader he admired, usually sat. They weren’t exactly neighbors, but their houses were close enough that Shiva felt a natural fondness for him. As the van rolled out of the parking lot and merged onto the highway, Shiva barely registered the traffic. His mind replayed the moment in science class again and again. Had he ruined his first impression? Did she think he was careless? Foolish? He stared out of the back window, watching other school vans pass by, buses packed with older students, and clusters of seniors riding bicycles with practiced ease. And then, suddenly, his gaze froze. Across the highway, walking alone with a handkerchief held over her face to block the dust and heat—was Gauri. Shiva’s spine stiffened. His breath caught. Their eyes met for the briefest heartbeat through the moving window. Just a second. Maybe less. But she looked at him. And he looked back. Then she lowered her head, and the moment slipped away as the van moved ahead. Shiva leaned back slowly, heat rushing to his cheeks, heart pounding like he had just won something enormous. He felt weightless. Over the moon. For the first time all day, he wasn’t thinking about homework or embarrassment or anything else. She had seen him. And that was enough.