\chapter{Liminal}
School had become the place Shiva waited for every day. It gave him an identity—topper, class monitor—and now, Gauri. He continued sitting on the second bench, directly behind her, and copying from her diary had quietly turned into a routine. “Hey… um…” he fumbled one day, hesitating long enough for both girls to turn toward him. “What was your name again?” he asked, cheeks warming. Both of them burst out laughing. “Gauri, you idiot,” Divya said, still chuckling. “How many times have you already asked?” “It just… slips from my head,” Shiva replied, tapping his forehead lightly. “Remember it properly this time,” Gauri said, a soft smile lifting her face. “If needed, write it a hundred times in a notebook—like kindergarten kids do to learn words.” That question had become his icebreaker ever since he’d started talking to her. “Sure,” he said, smiling back. “I’ll write it someday and never forget. I promise.”

Small talk about diaries, homework, and the occasional quiz had slowly become his moments with her. Strands of her hair still escaped her neatly tied ponytails and brushed against his notebook, but now he no longer pulled it away. Instead, it brought a strange, quiet happiness to his chest. He felt closer than ever. If the world were to stop at that moment, he thought, he wouldn’t mind being fixed there for eternity. Then came the ominous day. The school administration suddenly decided that girls had to sit in their own row. The small world Shiva had built for himself—two benches, first and second—shattered at once. But what could he do? Fate, he felt, was not something you could argue with. The girls were assigned the last row farthest from the door, right in front of the podium. Gauri chose a middle seat beside Divya. Shiva moved to the second row. And after that, seats hardly mattered anymore. By every period, he could be found on a different bench, beside a different classmate—a quiet wanderer in the small world he still called his own.

One day, nearly half the class was on its feet as Mrs. Uma stood at the front for her Sanskrit period. She was an old lady—perhaps in her sixties—her age etched clearly into her wrinkled face and frail, dry frame. Her voice, however, was as strict as discipline itself. She had a fierce devotion to her subject, arriving punctually every day and working with unwavering dedication despite her age. That morning, she was asking students to recite Sanskrit verb forms, homework she had assigned earlier, while simultaneously checking notebooks where the same forms were to be written. Shiva had done neither. His pages were blank, and so was his mind. As Mrs. Uma advanced bench by bench, his heart sank. There was no time to both memorize those merciless verbs and write them down. In panic, his eyes searched the room and fell on Gauri, sitting calmly at a distant bench. Slowly, carefully, he slipped into the seat of the adjacent row near her. “Hey, Gauri,” he whispered urgently, “can you please write my copy?” She glanced at him. “Why? Haven’t you done it yourself?” “I forgot,” he admitted. “Please help me now—otherwise she’ll definitely beat me with those sticks, and the scolding will be even worse.” She sighed, half-mocking yet kind. “Okay. Pass the copy.” “Thanks,” he murmured, handing the notebook over through Divya with a nervous smile. Returning to his seat, Shiva borrowed a Sanskrit book from the bench ahead and began chanting the verbs aloud under his breath, like someone performing a desperate ritual. Mrs. Uma moved through the class like a slow tsunami—one stick for every wrong verb, five for an incomplete notebook. As she drew closer, his chanting grew faster, almost frantic. “Take this,” Gauri whispered suddenly, holding out his notebook. “Thanks—you just saved me at least five sticks,” he whispered back, forcing a smile, and resumed chanting as if warding off a dark spirit. At last, Mrs. Uma stopped before him. “Shiva, show me your copy and tell me the verb forms.” He handed over the notebook, stood up, and recited in the same steady chorus. She examined the pages carefully, noting the neat, elegant Devanagari script, the beautifully curved letters. “Alright, sit down,” she said at last, closing the notebook and moving on. Shiva exhaled deeply. He was saved—and Gauri had saved him.

Then came the winters, and with them the midterm examinations. At the start of the academic year, each student had been given a workbook for every subject. Those books had remained untouched, their pages blank and pristine, forgotten at the bottom of schoolbags. Now, suddenly, the administration decided they must be put to use. Every workbook had to be completely filled. It was a ridiculous idea—one workbook per subject, each packed with pages. First, one had to search for the answers, and then painstakingly write them all down. Shiva got to work immediately. He finished his Mathematics workbook in just a couple of days. He was the first in the class to do so. Word spread quickly. One by one, classmates came to him asking for help, requesting to borrow the completed book. He handed it over to the one who came first, without much thought. Then Gauri came. “I heard you’ve completed the workbook?” she asked “Yes, I did,” he replied, his voice tinged with pride. “Can you lend it to me for one day? I’ll complete mine at home today,” she said politely, her eyes filled with hope. Trouble. He had already given the book to a friend. He couldn’t ask for it back just for Gauri—what would they think of him, abandoning friends for a girl? No, he couldn’t do that. Yet he couldn’t turn Gauri away either. Their friendship, though small and new, meant the world to him. “I don’t have it right now,” he said, the words heavy with anguish. “Oh, you already gave it to someone,” she replied, disappointment evident in her voice. He couldn’t let it end like that. She couldn’t struggle over something as simple as a workbook while he was there. “Why don’t you give your workbook to me?” he suggested. “I’ll fill it for you.” Her face lit up. “Really? You can do that for me?” she asked, smiling brightly.
“Why not? I can do anything for you—” he said quickly, then corrected himself, “I mean, friends help each other, right?”

He took her workbook and completed it in a single day — finding answers, solving questions, writing everything neatly. When he returned it, he asked for her other subjects too, and she gave them to him without hesitation. He completed all of her workbooks, while his own remained blank. Only after finishing hers did he begin working on his own. Days passed—perhaps a whole week. He wrote late into the night, during free periods, even through running classes, racing against time to meet the deadline. One day, he approached her again. “Gauri, I’ve completed most of my workbooks. Only two are left—English and Social Science. I’m already writing the Social Science one. Can you please write the English workbook this Sunday? If it’s done, I’ll meet Monday’s deadline.” “I can’t, Shiva” she turned her eyes away continuing “I already have a lot of work to do at home,” she replied sharply, before hurrying back to her seat.

He stood alone in the corridor, the pale winter sunlight falling on his woollen blazer. His eyes followed her as she walked back into the classroom, settling into her bench and talking to her friends. He remained there, silent and shaken, with the submission deadline drawing dangerously close.