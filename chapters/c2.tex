\chapter{First Interaction}
The second seat of the second row had quietly become Shiva’s place. Vijay followed him there without question—Vijay behind Divya, and Shiva directly behind Gauri. Yet Shiva still hadn’t spoken a single word to her. It surprised even him. He was usually effortless with people, the kind who could strike up a conversation with anyone. Talking to girls had never made him nervous. But this time, something invisible held him back. What would she think of me? The question circled endlessly in his head. Gauri carried herself differently—confident, composed, almost distant. She barely spoke, except to Divya beside her. Shiva watched her from behind, wondering how she would react if he suddenly spoke. What if she brushed him off? Worse—what if she smiled politely and turned away, as if he didn’t matter at all? And even if he did speak… what would he say? Thoughts crowded his mind, one after another, until he finally gave up and turned his attention to the blackboard, his books, his friends—anything that might quiet the noise inside his head. But Gauri refused to stay out of it. Sometimes, as she leaned forward during class, her ponytails would slip back and brush against the edge of his notebook. Each time it happened, his pen paused mid-word, his thoughts drifting helplessly toward her again. Sitting behind her had been a mistake, he decided. From another row, at least he could steal a glance—catch a glimpse of her face. Now, she was closer than ever, yet somehow farther away. The old line drifted into his mind without invitation: darkness beneath the lamp. It felt painfully true.

On that day, the Math teacher was absent.  On such days, it was customary for a replacement teacher to arrive—usually someone borrowed from another wing of the school. Until then, the class was left to its own devices. Shiva, the class monitor, took it upon himself to “manage” things. He walked up and down the aisles with an air of quiet importance. Not the tyrant kind of monitor—the ones who scribbled names on the blackboard and acted like miniature dictators—but the friendly sort. No threats. No punishments. No shouting. Students moved from bench to bench freely. Conversations floated across the room. Shiva himself had no real reason to be at half the desks he visited, but walking around felt right. After all, this was what being monitor looked like. Still, it wasn’t chaos. There was laughter, gossip, shifting seats—but no screaming, no desks overturned. Shiva’s only rule was simple: if a teacher walked in, the class should look like a class. And eventually, one did. She was from the junior wing—one of those teachers who taught kindergarten students, maybe up to third standard at most. Shiva noticed her just as she reached the doorway. Oh, no. Not her, he thought. The moment she stepped inside, the class snapped into silence. Students hurried back to their seats as if someone had pulled an invisible string. Shiva slid back into the third row, settling into the seat of the friend he had been talking to moments earlier. The room fell unnaturally quiet—like a tired animal collapsing after a long run. No one spoke. From beside Shiva, a boy leaned in and whispered, “Why don’t you ask her for a quiz during this free period?” Shiva sighed. “Do you really think she’ll agree?” “What else are we going to do for the whole period?” the boy pressed. Shiva glanced toward the teacher, then back at the class. Should I ask? he wondered. Quizzes were his strength. He liked them. He prepared for them. Not only that, but he and a few others had memorized almost the entire Lucent’s General Knowledge book—dates, facts, capitals—just in case moments like these appeared out of nowhere. And now, one had. Shiva straightened slightly, still unsure, but the thought had already rooted itself in his mind.

“Ma’am, can we have a quiz, please?” Shiva asked, his voice soft, almost pleading. The teacher paused and looked around, scanning the room as if trying to locate the disturbance. Shiva quickly stood up, saving her the trouble. “Ma’am, please,” he added. “We still have a lot of time.” She adjusted her spectacles, unimpressed. “I don’t think anyone would even join,” she said calmly. “Most of you are busy anyway. It’s better to do your homework while you have free time.” “But ma’am,” Shiva persisted, “we don’t have any homework. It’s only the second period. Many of us want to have a small round.” A sudden chorus erupted from Shiva’s row. “Yes, ma’am!” The teacher hesitated for a moment, then sighed lightly. “Alright,” she said. “Those who want a quiz—raise your hands.” Instantly, almost every boy shot his hand into the air. A few girls followed, glancing at one another before committing. Those already participating began nudging their neighbors, whispering encouragement, urging more hands up. Within seconds, the decision was obvious. The quiz was inevitable.

“No, ma’am. They’ll disturb us. I’m reading right now.” Gauri’s voice cut cleanly through the room. Shiva blinked, momentarily caught off guard. Reading? Right now? He glanced around—no books open, nothing urgent in sight. “But ma’am,” he said, turning back to the teacher, “everyone wants to play.” The teacher considered this, then spoke firmly. “Alright. Those who want to have a quiz may do so, but there must be no nuisance. Those who want to read will read. No one disturbs anyone.” “They always say that, ma’am,” Gauri replied at once. Her voice was steady, edged with resolve. “They make a lot of noise and disturb everyone.” A few girls nodded in agreement. “Stop her, Shiva,” the boy beside him whispered. “She won’t let us have even a small quiz. Who does she think she is?” Shiva exhaled. So be it, he thought. Duty to the comrades. “They’re just scared to lose, ma’am,” he said, his tone light but pointed. The words landed harder than he expected. Gauri straightened in her seat and turned halfway around. “Who do you think you are?” she shot back. “You act like you’re so smart—but you’re not.” The class fell silent. Shiva said nothing. Bragging had never been his way, and defending himself now felt unnecessary. He could feel eyes on him—from every corner of the room. “What is this behavior?” the teacher snapped. “What is your name?” “Gauri, ma’am,” she replied, her voice controlled but tight. “Gauri,” the teacher said, each syllable heavy with warning, “if you don’t want to join, you may continue your work. But this is not how you speak. Apologize to Shiva.” For a moment, Gauri stayed still. Then, without looking at him, she said softly, “Sorry.” Her face flushed red as she turned back to her desk and sat down, shoulders stiff with anger and frustration. The air in the classroom remained tense—charged, unsettled—as if something fragile had just been cracked.

The quiz commenced with the teacher pulling out a scrap of paper to keep score, dividing the class into two halves — two rows each—with Shiva and Gauri standing in opposite camps, though it hardly seemed at first that she intended to participate at all. Questions began flying back and forth, and Shiva answered many with ease, supported by the usual quiz enthusiasts, while his team’s secret weapon—a boy famously nicknamed Inventor for his uncanny memory of obscure innovations and their creators—left the opposing side helpless with questions about forgotten inventors of washing machines, automatic rifles, and countless other oddities, pushing the score steadily upward. Then, unexpectedly, Gauri stood up from the other side and asked, “What is the full form of ALU?” Shiva almost laughed to himself—wasn’t this taught in third grade?—and replied instantly, “Arithmetic and Logic Unit,” finishing with a faint, taunting smile as another point was added and Gauri sat down in silence, visibly embarrassed. By the time the bell rang, the gap was humiliating—the opposing team hadn’t even managed half of Shiva’s score—and though it was a clear victory, Shiva found no real joy in it, his thoughts already spiraling elsewhere: Was it worth it? Had he angered her beyond repair? Had he embarrassed her too much? The thrill of winning faded quickly, drowned beneath a growing guilt that refused to quiet down.

As usual, school ended, yet Shiva never found the courage to speak to her. Their very first interaction had gone so wrong that it only deepened the distance he already felt, and by the time he reached home, guilt and remorse clung to him like dust after a long walk. After lunch, he sat beside his Nani as the afternoon settled into the house. She broke the silence first. “Ramujagir’s tractor will come to our field today for ploughing. You go and keep an eye on it—these days they’ve grown lazy. A few rounds on the topsoil, corners untouched, and they leave.” Shiva asked quietly, “What time did he say?” “Five in the evening,” she replied. “Alright, I’ll go. Where is Mama? Isn’t he coming?” “No, he’s gone to town for some work,” she said, shaking her head. “People today don’t know how much hard work farming takes. Others in the village are growing vegetables, selling in the market, doing well. Your Nana managed a government job and still farmed. He would plough the fields before dawn with oxen and then leave for work.” Her voice softened into a sigh as they sat together in the hall, surrounded by fading portraits of Nana and cousins lining the walls, watching the afternoon drift past in quiet remembrance.

While sitting there, Shiva noticed two familiar figures pass along the road in front of his house—Rohan and Manoj, village friends and neighbors—waving exaggeratedly for him to join them. He got up, crossed the yard, and met them beneath a nearby bael tree. “Are you coming for the cricket match today?” Rohan asked. “I have to go to the fields,” Shiva replied briskly, leaning against the thin trunk. “Ramujagir is coming to plough.” Manoj plucked a leaf and began stripping its hard vein with his teeth. “When will you be back? You can join even if it’s late. We’ll play one or two matches till then.” “He told me to come at five,” Shiva said, sinking his weight further into the tree. “He’s ploughing Tedhiya—it’s not very big. Should take about an hour.” Rohan nodded. “Alright. We’re going to call the others and will leave by four. We’ll catch you there, then? At least come—don’t make excuses this time.” The three of them leaned against the tree, slowly curling around it like snakes around a sandalwood trunk, when a voice called out from the house. “Shiva!” “Coming, Nani,” he called back, straightening himself. He walked away with steady steps, leaving the tree quiet again in the afternoon heat. Back inside, he sat beside Nani once more, resting his head in her lap, his eyes drifting toward the clock. Her soft fingers moved gently through his hair, and a deep sense of peace settled over him. Slowly, his eyelids grew heavy, and he drifted into sleep.

“Bhaiya … bhaiya … wake up.” Rahul — Mama’s younger son — was shaking his arm gently, his voice tugging him out of sleep. Shiva stirred on the same sofa where he had dozed off earlier. From the inner gallery, where Nani sat on her small daybed, her voice followed, firm with concern. “It’s already five, Shiva. The tractor must have reached the fields by now.” He pushed himself upright, sleep still clinging to his eyes. “Bring the cycle,” he said calmly, and headed toward the backyard alley. After washing his face and having a cup of tea, he set off for the fields.

Returning home late from the cricket match with his friends, he had dinner and went straight to his bed. He opened his books, but his mind refused to stay with the pages, drifting instead into a tangle of thoughts. Sleep wouldn’t come. After a while, realizing he hadn’t read a single word, he pushed the books aside and lay back. His eyes settled on the ceiling fan above, slicing warm air through the room in slow, steady circles. Wide awake after a day crowded with events, he stared on, thoughts whispering endlessly. I shouldn’t talk to her. She doesn’t want to talk to me at all. I don’t want to upset her even more. Today was already enough. The fan kept turning, and his thoughts kept wandering.