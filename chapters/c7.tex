\chapter{Retort}
“I didn’t expect this to happen,” Suraj said, breaking the strange silence that had settled over recess. The whole group had gathered around a single bench, unusually subdued.
“Did you say something to her?” Aryan asked.

Shiva, weighed down by guilt and shame, couldn’t bring himself to answer. He kept his eyes lowered, fingers tracing invisible lines on the desk.

“You helped her every time,” Rajveer said, disgust sharp in his voice as his gaze flicked briefly toward the row where the girls sat. “And this is how she returned it. I told you—girls are no one’s friends. They only think about themselves. Always selfish.”

The words continued to fly around him, but Shiva barely followed them. He was still trapped in the memory of yesterday’s humiliation. Yet, amid the bitterness and anger, the sympathy he was suddenly receiving softened something inside him. It didn’t erase the shame—but it dulled its edge, if only a little.

“It doesn’t matter,” Shiva said at last, breaking his own silence. “Let her think whatever she wants.” He paused, drawing a slow breath, as if convincing himself. “I’m doing well in studies. I can talk to people easily. And…” he hesitated for a second, then added quietly, “I don’t even have an ugly face.” A strange calm settled over him as he finished, almost like relief. “If I ever wanted to,” he said, forcing a lightness into his voice, “I could talk to any girl.”

The bell rang, and everyone returned to their seats, leaving Shiva alone on the last bench with his thoughts. He sat like a statue through the next few periods, barely aware of the class unfolding in front of him. His attention drifted instead to the corridor, where the occasional passerby broke the stillness. Then his gaze settled on a pair of pigeons perched on the narrow ledge beyond the railing.

One of them would fly off now and then, returning with a twig clenched in its beak. The other waited, arranging each twig carefully into what could loosely be called a nest—a scattered, fragile pile. And every time the first pigeon returned, the other pecked at its feathers, impatient and insistent, until it flew off again. Shiva watched the quiet routine in silence, the small, relentless motions unfolding while the rest of the world seemed to move past him.

After school, he took his usual place in the van, settling into the back seat. His eyes stayed fixed on the familiar stretch of road across the highway, waiting. As the van rolled forward and bicycles passed in steady rhythm, he caught that brief patch of distance where he always saw her.

She was there—as always—with the handkerchief held to her face. But this time, she didn’t look up. Not even a passing glance. The moment slipped by quickly, swallowed by motion and noise.

He leaned back, a hollow settling inside him. His head felt strangely empty, and at times he couldn’t even tell what he was supposed to feel anymore.

He spoke to no one at home, heading straight to his room. Without changing out of his school uniform, he lay down on the bed, his eyes drifting up to the ceiling fan as it turned endlessly above him. Was he supposed to prove his feelings to her? he wondered. Did she even know what he felt—or who he was to her at all? Thoughts flooded his mind all at once, crowding one another until none of them made complete sense. Slowly, a quieter realization settled in. Maybe giving was the only true form of love. Maybe one shouldn’t expect anything in return.

Lying there, his gaze drifted toward the window, where the small razor blade rested—something he used for trimming his nails. The thought arrived suddenly, sharp and unwanted. For a moment, it scared him with its clarity.

He sat up at once, breath uneven, his heart pounding harder than before. A strange determination had flickered through him, only to dissolve just as quickly into something colder—fear, perhaps, or hesitation. He rubbed his forearms, as if trying to wipe the thought away, then pulled his woollen blazer off and folded it beside him.

Sitting there with his sleeves rolled up, he stared at his hands for a long time.

Much later, he sat still on the bed, staring at his arm as if it no longer belonged to him. The letters he had carved burned into his vision even when he closed his eyes.

He lay back again, staring at the ceiling fan as it continued to turn—steady, indifferent, unchanged. He didn’t know when his eyes finally closed, or when sleep took him.

A voice woke him, calling from the other side of the door. His eyes fell instinctively to his arm. Whatever he saw there made his chest tighten. He pulled his sleeve down quickly and got up to open the door.

“You’re still in your school uniform,” Mami said, standing in the doorway.

“Yeah,” he replied with a tired yawn. “I was exhausted. Fell asleep as soon as I got home.”

“Change and freshen up. Everyone’s waiting for you for dinner,” she said calmly.

“Okay. I’ll come.”

He changed quickly, choosing a long-sleeved shirt and a jacket, then rushed to the bathroom. Whatever had happened was his alone. He couldn’t let anyone see it.