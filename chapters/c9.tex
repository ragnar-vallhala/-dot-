\chapter{Refraction}
April’s warmth had long surrendered to November’s cold. Each morning, the grass lay heavy with dew, glinting like a field strewn with pearls. Shiva woke early that day. Beyond the window, the sun was just beginning to cut through the thick veil of fog, its light hesitant, diluted.

The late-night online conversations had left him drained, his eyes aching, but not enough to dull the quiet anticipation growing inside him. Today, they were leaving for a family event in Ayodhya—everyone together. And Shreya would be there.

He finished getting ready long before everyone else and spent the time waiting in the hall, moving things around, packing small items whenever someone asked. Eventually, he settled onto the couch in the porch.

His gaze lifted when he noticed a small group—three or four people—approaching from the road. He stood up instinctively, squinting for a clearer view. She was there, walking slightly behind the others. A green lehenga flowed around her, large earrings catching the light with every step, her thick hair gathered high, worn like an unspoken crown.

He stepped forward as the elders arrived, bending to touch their feet. In between greetings and blessings, his eyes found hers for a brief moment. They exchanged a quiet smile—nothing more, nothing less—and it stayed with him.

Everything was finally ready — bags packed, children dressed, vehicles lined up outside and waiting. Shiva moved back and forth with Mama, loading the last of the items into the vehicles. Mama wasn’t coming along; he would stay back at home while the others left.

“ Don’t take too much time,” Mama said, wiping his hands on his kurta. “Leave early, or you’ll struggle to find space for the rituals at the temple.”

The elders boarded first, settling into their seats with practiced ease, young children lifted onto laps wherever space allowed. Shreya took one of the seats among them. As always, the teenage boys were left with the unspoken duty of adjustment. Shiva and his cousins climbed into the narrow back seats, knees pressed together, bodies folded into whatever gaps remained.

From there, he could see little of the road ahead—no fields slipping past, no changing sky—only the backs of seats and the quiet awareness that the journey had begun.

When they arrived, Shiva was met with the same familiar sight—narrow lanes packed with people, crowded markets humming with noise and movement. For a moment, it felt no different from home, as though the journey had folded back on itself.

After visiting the Hanuman Garhi temple, preparations for cooking began, as expected. Shreya joined the women, taking charge of kneading the dough, seated beside Nani. Their hands moved in quiet rhythm, voices low and steady.

Shiva, meanwhile, was handed the task of managing the children. He sat opposite the group, keeping an eye on restless limbs and wandering attention, listening from afar to the soft murmur of conversation drifting across.

He handled the role well. For some children, he had mobile games ready; for others, stories pulled from memory; and for the one persistently mischievous outlier, a single stern look that worked better than words. He sat with two boys close to his age—Shreya’s younger brother and his own brother. The three drifted easily from one topic to another: schools, the long journey, and even the contested temple in Ayodhya, discussed with the half-formed certainty of youth.

Now and then, Shiva’s eyes wandered toward Shreya. She was shaping puris, passing them to Shiva’s mother, who stood frying them in a wide pan. The rhythm between them was effortless, practiced. Watching them, a quiet thought rose in him: Isn’t this happiness? Everything felt sorted. No urgency. No fear. Just people doing what needed to be done.

The night before, he had spoken to Shreya online, his words full of anticipation for the day ahead. Now, sitting here, the calm of the moment wrapped around him completely. Whenever Shreya smiled at something his mother said—though he couldn’t hear the words—he found himself smiling too, without needing to know why.

After the cooking was done, everyone settled down for lunch. Distributing the food fell to the boys, with Shiva quietly taking the lead. He organized them without fuss—one passing out puris, another serving sabji, one carrying water, and one handling the sweets. The system worked smoothly.

Shiva carried a plate over to Shreya, who looked a little tired now, the earlier rhythm finally catching up with her. “Did my mother make you work too much?” he asked, handing her the plate with a smile.

“No,” she replied, smiling back. “I wish I could do more for her.” She paused, then added lightly, “That depends on you, you know.”

He had not words, just a smile was his answer.

He took the two boys close to his age to the nearby temple for blessings. The crowd there was unbearable—people packed tightly on the steps, bodies pressing forward inch by inch, police guards stationed at regular intervals, watching with practiced indifference. Movement came only in slow waves, each step earned with effort.

Beside the crowded stairway ran another path, almost empty. A few people appeared through it now and then, passing easily while Shiva struggled just to keep his footing in the human tide. For a moment, he wondered why he hadn’t taken that route, and why no one else seemed to notice it.

Then his eyes fell on the board standing quietly at its entrance.

VIP Entry Only.

The answer settled in without surprise. He had heard of such things before, spoken casually, almost jokingly—but seeing the divide for the first time made it real. Two paths leading to the same place, yet not meant for the same people.

He stood there, pressed among strangers, and wondered quietly what kind of god waited inside—one who could be reached faster through influence, through money slipped unseen. If divinity was the same for all, why did access feel measured, rationed, priced?

The thought lingered even as the crowd pushed forward, slow and patient, as if everyone had already learned which path was meant for them.

The crowd pushed on, carrying Shiva and the boys with it—through the main hall, past the sanctum, and then out again—without pause or choice. There was no moment to stop, no time to look back, only the steady forward pull of bodies and breath, until they were suddenly outside, the noise thinning as quickly as it had gathered.

He returned to where the others were seated, choosing a bench just outside the hall where the cooking was still going on. The air there was quieter. He sat facing the open side, his eyes resting on the Sarayu flowing at a short distance, calm and steady.

“Hey…”

He turned to see Shreya standing behind him. Without looking directly at her, he brushed his hand lightly over the empty space beside him, an unspoken invitation. She sat down, and he shifted his gaze back to the river.

“Why do I always find you alone?” she said, smiling. “Rooftops, empty benches—places where no one else sits. Either you’ll become a philosopher one day, or a madman. I’m not sure which.”

“It’s nothing,” he replied calmly, turning toward her now. “I just like the peace you find away from the noise this world keeps making.”

“Oh,” she said thoughtfully. “Then I’ve brought something that might be useful for you.”

“What?” Surprise flickered across his face.

She reached into her pocket and took out a small locket. “It has Lord Shiva’s image in it,” she said. “He might help you find peace in this world. I bought it from the nearby market—I went there with Didi.”

“Oh… thank you,” he said, taking it carefully. Inside the glass pendant, Lord Shiva sat in meditation, still and distant, untouched by motion.

“I guess he’s your favorite god,” she added.

“Yes,” Shiva replied quietly. “He belongs to this world, yet he’s beyond it.” He smiled faintly. “Even my name comes from him.”

A voice called out from the hall—his mother was calling him for lunch. Shiva rose slowly and met Shreya’s eyes.

“Thank you,” he said softly. “You understand me better than anyone.” He closed his fingers around the locket. “I’ll always keep this with me.”

Then he turned toward the hall, slipping it carefully into the pocket of his jeans as he walked away.

They reached home only after night had fully settled in. Shiva, Mama, and two others close to his age began unloading the vehicles, carrying bags and bundles inside one by one. From the kitchen, the familiar sound of cups being set down announced that Mami had prepared tea for everyone.

“Why don’t you stay the night?” Nani said to Shreya’s elder brother, who had been driving. “It’s already dark. You can leave in the morning.”

“No, no, Amma,” he replied gently, bringing his hands together in respect. “I have coaching in the morning. It will only take us about an hour to reach home from here. I missed today’s class, but if we leave now, I can still attend tomorrow’s.” His tone was polite, firm without being dismissive—decided, yet respectful.

Soon, Shreya appeared with a tray in her hands, cups of tea neatly lined up. She moved from one elder to another, placing each cup carefully.

“You’ve forgotten one for Shiva,” Nani said, her voice laced with gentle teasing.

“I must have missed the count,” Shreya replied, a little embarrassed.

“That’s alright,” Nani continued, waving it off. “Shiva, take one from the kitchen.” Then, taking a sip of her tea, she added casually, glancing toward Shreya’s elder brother, “You know, Shiva makes very good tea.”

Shiva followed Shreya through the gallery. They passed the kitchen, climbed the stairs, and stepped onto the rooftop —the same porch where their friendship had quietly begun that night.

Their eyes moved instinctively toward the neighboring roof, the one that had once held the man and his careful rituals. It was empty now. Only the bulb remained, glowing faintly from the bamboo pole, swaying ever so slightly in the night air.

They stood there for a while without speaking. Their gazes met briefly, then drifted toward the distant bulb glowing in the dark. A cold breeze moved past them, unnoticed, as if it had no claim over that moment.

“You know,” Shreya said at last, breaking the silence, “I wanted to tell you something.” “Yes?” Shiva replied softly.

She took a breath. “I think… I love you.” The words came out carefully, but her heart was racing, loud enough that she was sure it would give her away. “Do you… feel the same?”

Shiva smiled, the kind that arrived before words. “I think I knew it already,” he said lightly. “A watchman can’t stay alone forever. A watch-woman is necessary, after all.”

Then, more quietly, without the joke to shield him, he added, “Of course I do.”

A call from below broke the moment Shiva wished could last forever. Shreya’s family was getting ready to leave. She hurried down the stairs, her steps quick, reluctant. Shiva remained on the rooftop, standing in the cold breeze, long after the sounds of departure faded. The moon slipped behind a veil of fog, and the night grew quieter.

Then his phone rang.

It was Shreya, calling from home.

They spoke through the night—slowly, eagerly— revisiting every glance, every word, every small moment of the day they had shared, as if saying it aloud might keep it alive just a little longer.

He returned to school, and sometimes his eyes would catch Gauri for a fleeting moment. His heart still stirred at the sight of her, a quiet ache lingering—questions rising uninvited about why she had done what she did, what she might have felt then. But those questions no longer held the same weight. They surfaced, hovered briefly, and passed.

Half of eighth standard had already slipped by. It had been nearly a year since he last spoke to her. Sometimes, he would still spot her on the far side of the highway while returning home, though even that became rarer with time. He had shifted his seat in the school van to the first row beside the driver. Krishna, the senior he once admired, had started coming on a bicycle, and Shiva had unknowingly become the most senior among those who rode the van now. The dusty back seat no longer interested him.

Without quite realizing it, he had moved forward—closer to the front, closer to himself—leaving certain questions behind, unanswered but no longer urgent.