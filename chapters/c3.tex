\chapter{Introspection}
Shiva came to class and took his usual place—the second bench in the second row—but this time he was unusually quiet. He didn’t speak to Vijay or Rajveer, not even in passing. Ms. Sneha went over the answers from the chapter they had just finished, and he copied them down without a word. The loose strands from Gauri’s ponytails still brushed the edge of his notebook now and then, but each time, he shifted it away, creating a careful distance. The period dragged on, and when the bell finally announced recess, Shiva slipped out of the classroom without waiting for anyone, sidestepping the familiar rush and noise, and stopped in the corridor. After a while, Divya joined him. “What happened?” she asked. “You’re unusually quiet today.” “Nothing… nothing,” Shiva replied quickly. “It was just a bit hot inside, so I came out for some air.” “Oh, don’t hide,” Divya said gently. “I know everything.” “What everything? I don’t know what you’re talking about,” he said, forcing a faint smile. “Believe it or not, I understand,” she continued. “Just don’t take it too seriously. She’s always a little rude.” Shiva gripped the corridor railing, warm from the sun. “She isn’t angry with you at all,” Divya added. Just then, Rajveer walked up. “Why are you two standing here?” he asked. “Nothing,” Divya replied softly, a sigh slipping into her voice. “Just waiting for recess to end.” “Isn’t today the selection for the morning assembly prayer group?” Rajveer asked. “Yes,” Divya said, her eyes lighting up. “I almost forgot. I think Gauri will join it.” Without saying anything, Shiva turned back toward the classroom, his head lowered as he returned to his seat.

Then came the uneasy moments before the final bell—the time when the class teacher arrived to sign everyone’s homework diary. Shiva’s heart sank. He hadn’t written homework for a single subject, yet the signature was unavoidable. He leaned toward Vijay. “Have you written your homework in the diary?” he asked quickly. “No,” Vijay replied with a grin, nodding toward Divya, who was busy filling out both her own diary and his. “Oh. Great,” Shiva muttered, defeated. Divya soon stood up and walked to the podium with both diaries in hand. Shiva hesitated, then glanced toward Gauri. She was already absorbed in a book. For a moment, he almost gave up—but then, summoning what little courage he had, he raised his trembling hand and gently tapped her shoulder. She turned sharply, surprise flashing across her eyes as she looked at him. “Have you completed your diary today?” he asked softly. Without a word, she slid her diary onto his desk. “Yes… return it quickly,” she said. “Sure. Thank you,” Shiva replied, picking it up at once. He began copying carefully, letter by letter, as if precision might somehow steady his nerves. A few minutes later, he placed the diary back on her desk. “Thank you,” he said, then added, almost involuntarily, “Your handwriting is really nice.” She paused. “No one says that. You might be the first,” she replied. “No, I mean the way you write ‘S’” he said with a small smile. “It’s very curved.” “Oh… thanks,” she said, already turning back to her book, the moment slipping away as quietly as it had come.

He boarded his van as he always did. Across the highway, Gauri walked on the opposite side, a handkerchief held lightly to her face—a sight that had quietly turned into a ritual. She never really looked at him, perhaps a passing glance on rare days, nothing more. Still, staying at home had begun to feel unbearable. Her presence lingered in his thoughts long after she disappeared from view.

Back home, he ate lunch and slept through the afternoon, exhaustion mixing with restlessness. As the sun began to soften, he grabbed his bicycle, slipped his Micromax keypad phone and earphones into his pocket, and headed out, casually mentioning at home that he was going to check on the crops—crops that didn’t really exist. The countryside, as always, held its quiet treasures: peace suspended in the air, unbroken and generous.

He left his bicycle beneath a tree in a bagh, a small orchard with only a handful of mango trees and a dense stand of bamboo along one edge. It lay far from both village and home—the nearest house more than two kilometers away. In the distance, the sun sank slowly, painting the sky in deep reds and spreading its glow across the trees. Below his fields stretched a dry riverbed, bare as bone for now, waiting for the monsoon to swell it into a river that would feel like a sea in its own right. Occasionally, herds of cows and buffaloes emerged on the horizon, moving lazily across the land. Everything felt quiet—serene, almost unreal—peaceful in a way that felt nothing short of magical.

How vast this world is, he thought. In all his life, he had barely stepped beyond the familiar—just the stretch between his own home and his mama’s village, seventy or eighty kilometers apart, a place he had known since childhood. Two places. That was all. What lay beyond that—what was hidden inside those distant villages? How would it feel to wander through their narrow alleys, to live a life entirely unknown to him? His gaze settled on a tiny cluster of houses across the riverbank, no more than specks against the fading light, and his thoughts drifted with them.

Then, inevitably, his mind circled back to Gauri. Does it really matter if she talks to me or not? he wondered. And even if she did—what then? Shouldn’t I be focusing on myself? One day, he would see the whole world. He knew that. Why should he anchor himself here, weighed down by someone else’s opinions, when there was so much waiting beyond this river, beyond this village, beyond himself?