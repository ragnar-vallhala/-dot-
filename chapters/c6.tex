\chapter{Off Guard}
Sun was dimly lit in those winters, surrendering the sky to cold gusts and a creeping fog, everyone in class seemed to notice the heat simmering inside Shiva. His gaze lingered too long, too often, and the obviousness of his interest in Gauri was hard to miss. Rajveer occasionally teased him, calling out her name, and Shiva would brush it off with a shy, blushing smile, though his heart raced all the same. Even the smallest interactions with her, brief and fleeting, lifted his spirits for days afterward.

One day, during arts class, he had sketched a portrait of her. Carefully capturing the glasses perched on her nose, the small scar on one cheek, the neat ponytails, every detail from memory, he had shown it to Divya to pass it along to Gauri. He didn’t dare hand it to her himself. When the sketch finally drifted into her hands, his gaze fixed on her face, searching for a reaction. Divya giggled as she handed it over, likely teasing Gauri. Hours of his previous night had gone into creating that sketch, and now he waited, breath held, for some sign—any sign—of appreciation. Gauri, however, cast only a brief glance at it before setting it aside, her expression neutral, unreadable. His hopes crumbled silently. Softly, he retrieved the sketchbook from Divya as she returned it, hiding the disappointment swelling inside him.

His mind was tangled in a strange, unresolved puzzle. Had he offended her? Was it too much? He drifted through the classroom, speaking to no one, lost in his own thoughts. When recess arrived, he stepped outside, letting the warm sunlight hit his face, but even there he couldn’t escape the maze in his head. Finding a quiet corner in the playground, he sat alone, trying to make sense of it all.

When recess ended, he returned to class, only to see her at her desk, head bowed low on the bench. Was she crying? His chest tightened, sinking deeper than he thought possible. He wanted to approach her, to ask what had happened, but fear rooted him in place. The room felt colder than ever, the walls closing in, shrinking him into his seat.

Soon it was time for dance class—the one period they had each week that he genuinely looked forward to. He joined his friends as the class moved to the dance hall. He knew nothing about dancing, didn’t even have the courage to perform a single step, but he enjoyed the lightness of the period, the freedom it brought. He settled in a corner at the back, paying no attention to the steps, his eyes constantly searching for her, silently urging himself to go and ask what had happened.

The dance period ended, and still he hadn’t seen her anywhere. Students filed back into the classroom, and he lingered near the taps, drinking water to delay returning. How could he tell her he was truly sorry? How could he promise he wouldn’t draw her again? The empty corridor seemed to stretch endlessly before him. Only after a long while, when the corridor was completely deserted, did he finally turn toward the stairs and make his way back to class.

An unexpected scene unfolded as he reached the corridor outside his classroom. Several students were standing there, clustered near the door, unusually silent, all of them staring toward the entrance as if afraid to step inside. Something was clearly wrong. He moved closer and peered in through the doorway.

Ms. Sneha was in the classroom—as expected, it was the Social Science period—but she wasn’t teaching. Her face was tight with anger, sharper than he had ever seen it. Gauri sat at her bench, her head lowered onto the desk, unmoving. Ms. Sneha was addressing the class in a raised voice, her words cutting through the room, though not all of them reached clearly into the corridor.

After a moment, her gaze shifted toward the doorway, where the boys stood frozen. She signaled them to come in. One by one, they entered the classroom, slow and cautious, forming a quiet line. Shiva joined them at the back, his steps hesitant, his eyes drawn once more to Gauri’s bowed head.

“It has been more than ten minutes since the class began. Do you people have any respect for time?” Ms. Sneha said, her voice sharp with anger, as she brought the stick down hard on the outstretched palm of the first student. He had already opened his hand, waiting for the inevitable. His face went pale, and his hand jerked back instantly, as if he had touched a naked electric wire.

She moved down the line without pause—two strikes for each student. The classroom was so silent that even a pencil dropping on the last bench would have echoed to the front.

So this is the day I get it too, Shiva thought, remembering how affectionate Ms. Sneha had always been toward him since the day she joined the school. His fingers, which had been tapping nervously against his thigh, slowly came to a halt. His eyes drifted away from the scene unfolding before him and fell on Gauri. She was still in the same position—head bowed, unmoving. Suddenly, the stick no longer mattered. The line shortened, one student after another stepping aside, until Shiva found himself standing right in front of her. Unexpectedly, Ms. Sneha placed the stick down on the podium beside her.

“I thought you were a good kid,” she said, her voice sharp and raised. “I never imagined you would do something like this.” Shiva felt something inside him collapse. Nothing made sense anymore. The entire class was staring at him—boys, girls, his friends, everyone. Yet he dared not speak a single word. He stood stiff, his legs trembling under his own weight. “There are people you think are sheep,” she continued, stepping aside and turning to face the class, “but in reality, they are wolves hiding under a sheep’s skin.”

Shiva’s eyes drifted to Gauri. She was still in the same position, unmoving, like a frozen rock. This was the first time he was being humiliated—publicly, mercilessly. How will I face any of them now? His thoughts raced wildly as Ms. Sneha’s voice kept pouring over him. “This is the age to study and build your future,” she went on, now sitting on the bench kept in front of the class, “not to indulge in such petty acts of love.”

Now, slowly, painfully, Shiva began to understand why he was being punished. “And just because you are good at studies,” her voice echoed, “does not make you Brahma. Life is long.” Sweat broke across his forehead despite the winter chill — an irony too cruel to notice. He stood frozen, unable to move or speak, his thoughts crashing over one another, drowning him completely. He could hear his own heartbeat, loud and violent inside his chest. If only the ground would open and swallow me, he thought.

Gradually, even his thoughts grew so loud that he could no longer hear her words. His vision blurred, the classroom fading, as if he himself was no longer there.

The moment stretched until the bell rang. Ms. Sneha stood up, gathered her books, and walked out. Shiva moved slowly back to his seat without looking at anyone and sat down. He did not remember how he reached home that day.