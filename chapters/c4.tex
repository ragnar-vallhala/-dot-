\chapter{Outcast}
It was a gentle morning. The sun was still rising, far from its full strength, and the sky lay heavy with scattered dark clouds. The backyard was damp, the narrow alley slick with moisture—clear signs that it had rained during the night. Shiva remembered half-waking to the sound of rainfall tapping against the world beyond his window, then drifting back into sleep. He rose early, bathed, and stepped into the veranda with a cup of tea warming his hands. Nani was already there, seated on her low stool, laying out clothes for her bath. “Giridhar, won’t be coming today to deliver milk,” she said softly. “His son will be at home instead. Someone needs to go and collect it.” Shiva took a careful sip of the hot tea. “It’s still more than an hour before the school van comes,” Nani continued. “Finish your tea and go get the milk.” “Okay… I’ll go,” Shiva said, then hesitated. “But I don’t know where his house is.” “It’s near the Samaya Mata temple,” she replied, adjusting her saree. “Just go there and ask anyone for Giridhar’s house. They’ll show you.” “How much milk should I take?” he asked, sitting down on the sofa and picking up yesterday’s newspaper from the side table. “Two liters, my son. Don’t you remember?” she said, glancing at him while holding the freshly ironed saree. “Take the vessel—he’ll measure it exactly. And tell him the milk has been thinner since last month.” “Sure,” Shiva nodded. “And don’t drink just tea,” she added firmly, concern slipping into her voice. “Take some namkeen from the almirah.” She paused, looking at the saree again. “You ironed this nicely last night. See how smooth the edges are.” A faint smile crossed her face. “If my legs hadn’t failed me, I’d still be ironing everything myself—mine and your Nana’s. Even back then, with that coal iron.” “What are you worrying about while I’m here?” Shiva said lightly, smiling as he flipped the newspaper to the global news page. “I can iron all your clothes in an hour now. I’m much faster than when I started.”

He got out as soon as he finished the tea. RCC roads had been laid throughout the village, but mud would still find its way during the rain. He crossed the main street and turned twice on narrower and narrower lanes. At one point he came across a cow being bound to a pole just adjacent to the narrow lane hardly a meter and half wide. Dung was all over the place. Rain had made everything worse, the cow probably moved over it and converted it in a thick slurry spread all over mixed with mud. Shiva was in his slippers. He can't get through this he thought, dirt will stick through all his slipper and foot. He paused for a moment to look nearby for maybe a way around this. Someone must have crossed this mess before him, their trail should be there. While still, he was searching a man arrived from the other side, staring at Shiva from far while approaching. He just came and crossed through the mud with no issues. Shiva felt embarrassed. Probably he might be thinking of him as some kind of city brat with foolish tantrums, he thought to himself. Without paying much attention he walked straight through it all. Mix of mud and dung got all over his feet, even walking became difficult with those slippery slippers. But at last, passing all the hurdles he reached the temple mentioned by Nani. It was on the eastern edge of the village followed by the riverbed and farms stretching through the horizon after that. Small traces of water was visible in the distant probably riverbed has started to gain water.

Looking around, he saw women busy with cattle, some tending to cows, others herding goats. On a few porches, girls—about his age or younger—were crouched near mud hearths, preparing for the day’s first meal. Thin streams of smoke already curled up from several houses, dissolving into the damp morning air. It felt different from his own home. They had a mud hearth too, but it was lit only on festivals; on ordinary days, the gas stove did the work. Here, everything ran on routine and necessity. Everyone moved with purpose, like parts of a well-worn machine, each absorbed in their task. He didn’t recognize a single face. A quiet hesitation crept through him. Whom should he ask? What if he was standing in the wrong part of the village altogether? The thought of asking for Giridhar—and being laughed at for not knowing his way around the very village he had grown up in—made his steps slow, his confidence thin.

``Aren't you Raghu's nephew?'' a female voice came from behind. He turned sharply to see a woman holding a worn out plastic tub against her waist filled with cattle fodder. ``Yes... I am.'' he felt like seeing this woman for the first time in his life. But still somehow she recognizes him. ``What are you doing here?'' she asked politely while her gaze fell on the handled vessel in his hand. ``So, you came for milk.'' she continued. ``Are you receiving milk from Giridhar?'' she asked while pulling the sliding tub up on her waist. ``Yes, I came for his house.'' he replied with gratitude and gentleness. She pointed towards a group of houses barely fifty meters away saying ``Go to that green house with the cow next to it. That's Giridhar's house.'' ``Okay... Okay aunty'' he replied while continuing to drag his slippery slippers on the muddy lane.

He reached the house quickly. Two adult men were seated on a cot nearby. “You got quite late—but at least you came,” one of them remarked, stretching out his arm to take the vessel from Shiva’s hand. “It’s very muddy in the narrow lanes,” Shiva replied, offering it as a reasonable excuse. “Which way did you come from?” the man asked, a note of concern entering his voice. “Did you pass through the lane near Rambihari’s shop?” “Yes, that way,” Shiva answered plainly. “You should’ve turned at the second cut from the main street. The road there is cleaner, fewer turns,” the man said, then glanced toward a young girl wiping chairs on the veranda. “These goat herders have ruined that lane—nothing but filth all the way.” Shiva stood quietly, unsure how to respond. “How much should I measure?” the girl asked, taking the vessel. “Two liters,” the man replied. She carried it inside. “Sit,” the man said, shifting to clear space on the cot. “Which class are you in?” “Seventh,” Shiva replied, lowering himself slowly onto the edge. “Good. Study with dedication and hard work—you’ll make all of us proud one day.” Shiva offered a faint, uncertain smile. “My son studies too,” the man continued, pride lighting his eyes. “Second class, MKS School. Very good at English. I don’t understand a single word, but he reads fluently—or so it seems to me. The school isn’t like your English-medium one, but I’m doing my best. I’ll support him for as long as he wants to study.” Shiva listened in silence, still as stone. The girl returned with the vessel filled with milk and handed it to him. Shiva stood up and said, sincerity softening his voice, “Alright, uncle. I’ll come again sometime. I have to catch my school van.” He walked away quietly, the weight of the conversation lingering long after the sound of his footsteps faded.

He returned home, placed the milk vessel in the kitchen. From there, he went straight to his room and changed into his school uniform. A few minutes later, he came back to the kitchen, quickly finished his breakfast, which Mami has prepared, and slung his bag over his shoulders before stepping out into the veranda. Nani was lighting incense sticks for the morning prayer, her hands moving slowly as thin streams of smoke rose toward the sun. “Why don’t you take your lunch?” she asked, her voice carrying both care and firmness. “I’ve already had breakfast,” Shiva replied, edging toward the exit. “I don’t feel hungry in class anyway. And the lunch break is very short.” He was about to leave when her voice grew sharper. “Do you think I don’t know how school works just because I didn’t go one myself?” she said. “Children take lunch every day. You’ll only burn your blood with hunger. At least take some biscuits—I’ve kept them on the table.” Shiva stopped, turned back at once, picked up the packet of biscuits, and hurried out to the road again.

Within moments, the school van appeared and came to a halt beside him. He climbed into his usual back seat and stared out quietly as his home slowly disappeared behind the turning wheels.